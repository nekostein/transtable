\documentclass[11pt,a4paper]{report}
\input{_dist/libvi/latex/ns-report.tex}
\usepackage{nicefrac}

\providecommand{\colorhref}[2]{\textcolor{black!20!blue!80!green}{\href{#1}{#2}}}
\hypersetup{pdftitle={TransTable Developer Documentation}}





\begin{document}
\nsreportcover{\today}{Neruthes}{TransTable\\Developer Documentation}{Nekostein HQ}{For Public Release}
\sffamily

\tableofcontents\clearpage





\chapter{Introduction}

\section{Using CSV for Internationalization}
In game development, we often use CSV to do internationalization (i18n).
For example, Godot official documentation
\footnote{Section ``Importing translations'', Godot Documentation 4.2.
(\href{https://docs.godotengine.org/en/stable/tutorials/assets\_pipeline/importing\_translations.html}{URL})}
has explained how CSV-based i18n works.

While Godot official documentation suggests that GNU gettext is a more powerful solution,
we at Nekostein often find CSV good enough.
Not only because it is simple and straightforward when editing,
but also because we use a simple shell script to extract
item descriptions for subsequent archiving workflows (PDF, wiki, etc) ---
if we were to use GNU gettext, this part of work would have been more complicated.

\section{Prioritization}
As a game developer, you may wonder in which regions your game will sell well.
If you were to make accurate speculations, your marketing strategy would have been easier,
and you would have been able to prioritize affording i18n for certain languages.

As of year 2024, ChatGPT and similar stuff are good at
providing translations in batch with acceptable quality.
Just tell ChatGPT to append another column for Français in the CSV
when you already have English and Deutsch for cross-reference.
We at Nekostein are bilingual in English and Mandarin
and we find this workflow functioning all right,
although we are not experts in Deutsch, Français, or Nihongo.

\section{Community Suggestions}
The quality of machine translations can be suspicious,
but we believe that they are helpful in starting a dialogue with players who use the language.
It is more convenient to propose a minor modification on the existing translation
than to start writing some translation.






\chapter{Workflow}

The workflow should be like this:
\begin{compactitem}
    \item Clone this repo.
    \item Configure in \texttt{config.json}.
    \item Put CSV files into \texttt{/files} (using shell scripts).
    \item Run build script \texttt{./make.sh www}.
    \item Get a static website in \texttt{/www}.
    \item Publish the static website.
    \item Ask a volunteer to edit in the webpage.
    \item The volunteer gives back a diff list.
    \item Paste some suggestions into the webpage.
    \item Get exported CSV text.
    \item Copy and paste to the local editor.
\end{compactitem}

The user need to do the publishing manually.







\chapter{Copyright}
TransTable is a free software published with GNU GPL 2.0.







\end{document}
